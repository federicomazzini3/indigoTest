\section{Requisiti}

\subsection{Requisiti di business}
Requirement di alto livello stabiliscono perché si sta facendo 
il sistema e quali siano i suoi vantaggi. 

\subsection{Requisiti utente}
Quando un utente usa il sistema, cosa vuole fare e cosa si aspetta?
Raccolgono le aspettative per un utente.
Un modo tipico per documentare i requisiti utente sono le user stories.

\subsection{Requisiti funzionali}
Funzionali: statement dettagliati delle funzionalità del sistema indicate
in modo chiaro (organizzarli permette uno sviluppo del progetto rigoroso)
\begin{enumerate}
    \item Generazione in modo casuale di una mappa 2D formata da più stanze
    \item Ogni stanza è generata in modo casuale e può essere di diversa tipologia:
        \begin{enumerate}
            \item stanza con 1 oggetto
            \item stanza con nemici ed elementi bloccanti
            \item stanza con boss (1 sola nella mappa)
        \end{enumerate}
    \item Personaggio con caratteristiche e punti vita
    \item Controllo del personaggio tramite movimento e sparo
    \item Diversi tipi di nemici con diverse tipologie di movimento: casuale o intelligente
    \item Diversi tipi di oggetti per potenziare le caratteristiche del personaggio
    \item Menù di gioco
\end{enumerate}

\begin{enumerate}
    \item Generazione di una mappa 2D in maniera procedurale formata da più stanze
    \begin{enumerate}
        \item Ogni stanza è generata in maniera procedurale
        \item Una stanza è di diverse tipologie
        \begin{enumerate}
            \item Stanza con 1 oggetto
            \item Stanza con nemici ed elementi bloccanti
            \item Stanza con boss (1 sola nella mappa)
        \end{enumerate}
            \item Una stanza ha due porte utilizzate per l'entrata e l'uscita del giocatore
            \begin{enumerate}
            \item All'entrata del giocatore nella stanza, le porte si chiudono
            \item Le porte di una stanza si aprono solamente quando il giocatore ha eliminato tutti i nemici all'interno
        \end{enumerate}
    \end{enumerate}
    \item Personaggio con caratteristiche e punti vita controllato dall'utente
    \begin{enumerate}
        \item Il personaggio è controllato dall'utente
        \begin{enumerate}
            \item L'utente può muovere il personaggio nelle quattro direzioni principali (sopra, sotto, destra, sinistra)
            \item L'utente può eseguire uno o più "shot" verso una delle quattro direzioni principali (sopra, sotto, destra, sinistra)
        \end{enumerate}
        \item Il personaggio può infliggere danno ai nemici
        \begin{enumerate}
            \item Il personaggio infligge danno ai nemici quando uno "shot" colpisce un nemico
        \end{enumerate}
        \item Il personaggio perde punti vita
        \begin{enumerate}
            \item Il personaggio è danneggiato, e i suoi punti vita diminuiscono, quando è toccato da un nemico
        \end{enumerate}
    \end{enumerate}
    \item Nemici
    \begin{enumerate}
        \item Un nemico può avere 2 tipologie di movimento
        \begin{enumerate}
            \item Nemico con movimenti casuali all'interno di una stanza
            \item Nemico con movimento in direzione del giocatore
        \end{enumerate}
        \item Un nemico è presente in una e una sola stanza, da cui non può uscire
        \item Un nemico è danneggiato, e i suoi punti vita diminuiscono, quando è colpito da uno shot del giocatore
    \end{enumerate}
\end{enumerate}


\subsection{Requisiti non funzionali}
Non funzionali: statement dettagliati che raccolgono la qualità del 
comportamento o sui vincoli del sistema

\subsection{Requisiti di implementazione}
Implementazione: requirement posti agli sviluppatori. Danno indicazioni 
sul lavoro da fare. Possono comprendere: tipi di strumenti utilizzati, 
modi di sviluppo, documentazione fornita.
