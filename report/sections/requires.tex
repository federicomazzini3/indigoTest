\section{Requisiti}

\subsection{Requisiti di business}
%Requirement di alto livello stabiliscono perché si sta facendo 
%il sistema e quali siano i suoi vantaggi. 
\begin{itemize}
    \item Creazione di un gioco di genere Roguelike su piattaforma JVM
    \item Possibilità di gioco su browser
\end{itemize}

\subsection{Requisiti utente}
%Quando un utente usa il sistema, cosa vuole fare e cosa si aspetta?
%Raccolgono le aspettative per un utente.
%Un modo tipico per documentare i requisiti utente sono le user stories.

Matteo Brocca in questo progetto ricopre il ruolo di esperto di dominio e committente. 
E' un appassionato di giochi Roguelike ma essendo troppo bravo li ha finiti tutti. 
Da qui l'idea di crearne uno nuovo per lui. 

\begin{enumerate}
    \item L'utente può avviare una nuova partita
    \begin{enumerate}
        \item All'inizio di ogni partita l'utente partirà con caratteristiche e punti vita base,
        all'interno di una mappa composta da stanze definite in modo procedurale
    \end{enumerate}
    \item L'utente potrà controllare un personaggio all'interno della mappa per
    \begin{enumerate}
        \item spostarsi all'interno della mappa
        \item sparare ai nemici
        \item raccogliere elementi utili all'aumento delle sue caratteristiche
    \end{enumerate}
\end{enumerate}

\subsection{Requisiti funzionali}
%Funzionali: statement dettagliati delle funzionalità del sistema indicate
%in modo chiaro (organizzarli permette uno sviluppo del progetto rigoroso)
\subsubsection{Requisiti funzionali 1.0}
\begin{enumerate}
    \item Generazione in modo casuale di una mappa 2D formata da più stanze
    \item Ogni stanza è generata in modo casuale e può essere di diversa tipologia:
        \begin{enumerate}
            \item stanza con 1 oggetto
            \item stanza con nemici ed elementi bloccanti
            \item stanza con boss (1 sola nella mappa)
        \end{enumerate}
    \item Personaggio con caratteristiche e punti vita
    \item Controllo del personaggio tramite movimento e sparo
    \item Diversi tipi di nemici con diverse tipologie di movimento: casuale o intelligente
    \item Diversi tipi di oggetti per potenziare le caratteristiche del personaggio
    \item Menù di gioco
\end{enumerate}

\subsubsection{Requisiti funzionali 2.0}
\begin{enumerate}
    \item Da un menu di gioco, deve essere possibile avviare una nuova partita
    \item Generazione di una mappa 2D in maniera procedurale
    \begin{enumerate}
        \item Una mappa è formata da più stanze
        \begin{enumerate}
            \item Una stanza è di diverse tipologie
            \begin{enumerate}
                \item Stanza con 1 oggetto
                \item Stanza con nemici ed elementi bloccanti
                \item Stanza con boss (1 sola nella mappa)
            \end{enumerate}
            \item Una stanza ha due porte 
            \begin{enumerate}
            \item La porta permette di entrare all'interno di una stanza
            \item La porta permette di uscire da una stanza
            \item All'entrata del giocatore nella stanza, le porte si chiudono
            \item Le porte di una stanza si aprono solamente quando il giocatore ha eliminato tutti i nemici all'interno
            \end{enumerate}
        \end{enumerate}
    \end{enumerate}
    \item Personaggio con caratteristiche e punti vita controllato dall'utente
    \begin{enumerate}
        \item Il personaggio è controllato dall'utente
        \begin{enumerate}
            \item L'utente può muovere il personaggio nelle quattro direzioni principali (sopra, sotto, destra, sinistra)
            \item L'utente può eseguire uno o più "shot" verso una delle quattro direzioni principali (sopra, sotto, destra, sinistra)
        \end{enumerate}
        \item Il personaggio può infliggere danno ai nemici
        \begin{enumerate}
            \item Il personaggio infligge danno ai nemici quando uno "shot" colpisce un nemico
        \end{enumerate}
        \item Il personaggio perde punti vita
        \begin{enumerate}
            \item Il personaggio è danneggiato, e i suoi punti vita diminuiscono, quando è toccato da un nemico
        \end{enumerate}
        \item Il personaggio può raccogliere oggetti che modificano le sue caratteristiche
    \end{enumerate}
    \item Presenza di nemici all'interno delle stanze
    \begin{enumerate}
        \item I nemici possono avere caratteristiche diverse
        \begin{enumerate}
            \item Un nemico può avere 2 tipologie di movimento
            \begin{enumerate}
                \item Nemico con movimenti casuali all'interno di una stanza
                \item Nemico con movimento in direzione del giocatore
            \end{enumerate}
            \item Un nemico può infliggere danno diverso in base alla sua tipologia
        \end{enumerate}
        \item Un nemico è presente in una e una sola stanza, da cui non può uscire
        \item Un nemico è danneggiato, e i suoi punti vita diminuiscono, quando è colpito da uno shot del giocatore
    \end{enumerate}
    \item Presenza di oggetti all'interno delle stanze
    \begin{enumerate}
        \item Un oggetto è un artefatto che modifica le caratteristiche di un personaggio
        \item Gli oggetti sono di diversi tipi e sono disposti in modo casuale all'interno della mappa
    \end{enumerate}
\end{enumerate}


\subsection{Requisiti non funzionali}
%Non funzionali: statement dettagliati che raccolgono la qualità del 
%comportamento o sui vincoli del sistema

\begin{enumerate}
    \item Fluidità di gioco
    \begin{enumerate}
        \item Il gioco non deve presentare lag o inestetismi marcati che rendano la user experience non ottimale
    \end{enumerate}
    \item Bilanciare numero e caratteristiche dei nemici per ottenere un livello di difficoltà di gioco medio
    \begin{enumerate}
        \item Il gioco non deve essere troppo facile 
        \begin{enumerate}
            \item Presenza di pochi nemici all'interno di una stanza
            \item Presenza di oggetti che potenzino le caratteristiche del giocatore rendendo il conflitto giocatore-nemici impari
        \end{enumerate}
        \item Il gioco non deve essere troppo difficile
        \begin{enumerate}
            \item Presenza di troppi nemici all'interno di una stanza
            \item Presenza di oggetti che non potenzino o riducano le caratteristiche del giocatore rendendo il conflitto giocatore-nemici impari
        \end{enumerate}
    \end{enumerate}
\end{enumerate}


\subsection{Requisiti di implementazione}
Implementazione: requirement posti agli sviluppatori. Danno indicazioni 
sul lavoro da fare. Possono comprendere: tipi di strumenti utilizzati, 
modi di sviluppo, documentazione fornita.
